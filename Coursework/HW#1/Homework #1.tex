
%%%%%%%%%%%%%%%%%%%%%%%%%%%%%%%%%%%%%%%%%%%%%%%%%%%%%%%%%%%%%%%%%%%%%%%%%%%%%%%%%%%%%%%
%%%%%%%%%%%%%%%%%%%%%%%%%%%%%%%%%%%%%%%%%%%%%%%%%%%%%%%%%%%%%%%%%%%%%%%%%%%%%%%%%%%%%%%
% 
% This top part of the document is called the 'preamble'.  Modify it with caution!
%
% The real document starts below where it says 'The main document starts here'.

\documentclass[12pt]{article}

\usepackage{amssymb,amsmath,amsthm}
\usepackage[top=1in, bottom=1in, left=1.25in, right=1.25in]{geometry}
\usepackage{fancyhdr}
\usepackage{enumerate}
\usepackage{listings}

% Comment the following line to use TeX's default font of Computer Modern.
\usepackage{times,txfonts}



\makeatletter
\renewcommand*\env@matrix[1][*\c@MaxMatrixCols c]{%
  \hskip -\arraycolsep
  \let\@ifnextchar\new@ifnextchar
  \array{#1}}
\makeatother

\newtheoremstyle{homework}% name of the style to be used
  {18pt}% measure of space to leave above the theorem. E.g.: 3pt
  {12pt}% measure of space to leave below the theorem. E.g.: 3pt
  {}% name of font to use in the body of the theorem
  {}% measure of space to indent
  {\bfseries}% name of head font
  {:}% punctuation between head and body
  {2ex}% space after theorem head; " " = normal interword space
  {}% Manually specify head
\theoremstyle{homework} 

% Set up an Exercise environment and a Solution label.
\newtheorem*{exercisecore}{Exercise \@currentlabel}
\newenvironment{exercise}[1]
{\def\@currentlabel{#1}\exercisecore}
{\endexercisecore}

\newcommand{\localhead}[1]{\par\smallskip\noindent\textbf{#1}\nobreak\\}%
\newcommand\solution{\localhead{Solution:}}

%%%%%%%%%%%%%%%%%%%%%%%%%%%%%%%%%%%%%%%%%%%%%%%%%%%%%%%%%%%%%%%%%%%%%%%%
%
% Stuff for getting the name/document date/title across the header
\makeatletter
\RequirePackage{fancyhdr}
\pagestyle{fancy}
\fancyfoot[C]{\ifnum \value{page} > 1\relax\thepage\fi}
\fancyhead[L]{\ifx\@doclabel\@empty\else\@doclabel\fi}
\fancyhead[C]{\ifx\@docdate\@empty\else\@docdate\fi}
\fancyhead[R]{\ifx\@docauthor\@empty\else\@docauthor\fi}
\headheight 15pt

\def\doclabel#1{\gdef\@doclabel{#1}}
\doclabel{Use {\tt\textbackslash doclabel\{MY LABEL\}}.}
\def\docdate#1{\gdef\@docdate{#1}}
\docdate{Use {\tt\textbackslash docdate\{MY DATE\}}.}
\def\docauthor#1{\gdef\@docauthor{#1}}
\docauthor{Use {\tt\textbackslash docauthor\{MY NAME\}}.}
\makeatother

% Shortcuts for blackboard bold number sets (reals, integers, etc.)
\newcommand{\Reals}{\ensuremath{\mathbb R}}
\newcommand{\Nats}{\ensuremath{\mathbb N}}
\newcommand{\Ints}{\ensuremath{\mathbb Z}}
\newcommand{\Rats}{\ensuremath{\mathbb Q}}
\newcommand{\Cplx}{\ensuremath{\mathbb C}}
%% Some equivalents that some people may prefer.
\let\RR\Reals
\let\NN\Nats
\let\II\Ints
\let\CC\Cplx

%%%%%%%%%%%%%%%%%%%%%%%%%%%%%%%%%%%%%%%%%%%%%%%%%%%%%%%%%%%%%%%%%%%%%%%%%%%%%%%%%%%%%%%
%%%%%%%%%%%%%%%%%%%%%%%%%%%%%%%%%%%%%%%%%%%%%%%%%%%%%%%%%%%%%%%%%%%%%%%%%%%%%%%%%%%%%%%
% 
% The main document start here.

% The following commands set up the material that appears in the header.
\doclabel{STAT 402: Homework 1}
\docauthor{Stefano Fochesatto}
\docdate{\today}

\begin{document}



\begin{exercise}{1} Suppose we know the values from the populations, 
    \begin{equation*}
    \begin{tabular}{c|c}
        Plot & Number of Weeds \\
        \hline
        1&4\\
        2&8\\
        3&5\\
        4&7\\
        5&5\\
        6&12
    \end{tabular}
\end{equation*}
Taking a sample consisting of plots 1, 3, 5 and 6,
\begin{enumerate}
    \item Compute the mean and standard deviation of the population. 
    Is this a parameter or a statistic?
    \solution Computing the mean of the population,
    \begin{equation*}
        \mu = \dfrac{4 + 8 + 5 + 7 + 5 + 12}{6} = \dfrac{41}{6}\approx 6.833
    \end{equation*}
    Computing the standard deviation of the population,
    \begin{equation*}
        \sigma =\left(\dfrac{(4 - \mu)^2+(8 - \mu)^2+(5 - \mu)^2+(7 - \mu)^2+(5 - \mu)^2+(12 - \mu)^2}{6}\right)^{\frac{1}{2}} = \left(\frac{257}{6}\right)^{\frac{1}{2}} \approx 2.62
    \end{equation*}
    These are parameters since they are computed with the entire population.
    \item Compute the mean and standard deviation of the sample. Is this a 
    parameter or a statistic?
    \solution Computing the mean of the sample,
    \begin{equation*}
        \bar{x} = \dfrac{4 + 5 + 5 + 12}{4} = \frac{13}{2} = 6.5
    \end{equation*}
    Computing the standard deviation,
    \begin{equation*}
        S =\left(\dfrac{(4 - \bar{x})^2+(5 - \bar{x})^2+(5 - \bar{x})^2+(12 - \bar{x})^2}{4}\right)^{\frac{1}{2}} = \left(\frac{41}{4}\right)^{\frac{1}{2}} \approx 3.69
    \end{equation*}
    These are statistics since they are computed using only a sample of the population. 
\end{enumerate}
\end{exercise}
\vspace{1in}

\begin{exercise}{2} What IS a sampling distribution?\\
    The sampling distribution of a statistic describes the probability of the values that that
    statistic can take on.
\end{exercise}
\vspace{1in}

\begin{exercise}{3} We are interested in the average number of trees on three small islands. Unknown to 
    us, the numbers are 20 trees, 30 trees, and 25 trees. We can only afford to visit two islands. We use a simple
    random sample without replacement, so that all samples are equally likely. \\
    \begin{enumerate}
        \item Write out all possible samples, 
        \begin{equation*}
            \begin{tabular}{c|c}
                Index & Sample \\
                \hline
                1&20, 30\\
                2&20, 25\\
                3&25, 30
            \end{tabular}
        \end{equation*}
        \item Find the sampling distribution of the average sample. 
        \begin{equation*}
            \begin{tabular}{c|c|c|c}
                Index & Sample & $\hat{x}$ & $P(\hat{X} = \hat{x})$ \\
                \hline
                1&20, 30 & 25&$1/3$\\
                2&20, 25 & 22.5 &$1/3$\\
                3&25, 30 & 27.5 &$1/3$
            \end{tabular}
        \end{equation*}
        \item Find the expected value of the sampling distribution, is this an unbiased estimator?
        \begin{equation*}
            E(\hat{X}) = \frac{1}{3}(25 + 22.5 + 27.5) = 25
        \end{equation*}
        \begin{equation*}
            \mu = \dfrac{20 + 25 + 30}{3} = 25
        \end{equation*}
        Since $E(\hat{X}) = \mu$ we know that $E(\hat{X})$ is an unbiased estimator. 
        \begin{equation*}
            V(\hat{X}) = \frac{1}{3}(25^2 + 22.5^2 + 27.5^2) - 25^2 \approx 4.166
        \end{equation*}
        \item What are two desirable properties in an estimator?\\
        A good estimator is unbiased, has the minimal variance, and high resistance to outliers. 
    \end{enumerate}
\end{exercise}
\vspace{1in}

\begin{exercise}{5}Suppose we have estimated moose populations on two islands.
    Each island was sampled independently.  The estimated count from the first island is $X = 120$ with a 
    standard deviation (standard error) equal to 10.  The estimated count from the second island is $Y = 200$ with a standard error of 40.  
    Find an estimate of the total moose on both islands, along with a standard error for this total. 
    Then create a 95 percent confidence interval for the total number of moose on both islands.\\
    \solution  Consider the following estimator $\tau$ for the total number of moose on both islands,
    \begin{equation*}
        E(\tau) = E(X + Y) = E(X) + E(Y) = 120 + 200 = 320 
    \end{equation*}
    Note that the total number of sampling units $N = 2$ so we get the following,
   Therefore we also know that, 
    \begin{equation*}
        V(\tau) = V(X + Y) = V(X) + V(Y) = 10^2 + 40^2 = 1700.
    \end{equation*}
    Computing the standard error we get that, 
    \begin{equation*}
        \tau_{S.E} = \frac{1700}{2}^{1/2} \approx 29.15
    \end{equation*}
    Finally we get a 95 percent confidence interval of 
    \begin{equation*}
        95_{CI} = (320 + 2(29.15), 320 - 2(29.15)) = (378.3, 261.7)
    \end{equation*}
\end{exercise}


\end{document}




















