
%%%%%%%%%%%%%%%%%%%%%%%%%%%%%%%%%%%%%%%%%%%%%%%%%%%%%%%%%%%%%%%%%%%%%%%%%%%%%%%%%%%%%%%
%%%%%%%%%%%%%%%%%%%%%%%%%%%%%%%%%%%%%%%%%%%%%%%%%%%%%%%%%%%%%%%%%%%%%%%%%%%%%%%%%%%%%%%
% 
% This top part of the document is called the 'preamble'.  Modify it with caution!
%
% The real document starts below where it says 'The main document starts here'.

\documentclass[12pt]{article}

\usepackage{amssymb,amsmath,amsthm}
\usepackage[top=1in, bottom=1in, left=1.25in, right=1.25in]{geometry}
\usepackage{fancyhdr}
\usepackage{enumerate}
\usepackage{listings}
\usepackage{graphicx}
\usepackage{float}
% Comment the following line to use TeX's default font of Computer Modern.

\usepackage{times,txfonts}



\makeatletter
\renewcommand*\env@matrix[1][*\c@MaxMatrixCols c]{%
  \hskip -\arraycolsep
  \let\@ifnextchar\new@ifnextchar
  \array{#1}}
\makeatother

\newtheoremstyle{homework}% name of the style to be used
  {18pt}% measure of space to leave above the theorem. E.g.: 3pt
  {12pt}% measure of space to leave below the theorem. E.g.: 3pt
  {}% name of font to use in the body of the theorem
  {}% measure of space to indent
  {\bfseries}% name of head font
  {:}% punctuation between head and body
  {2ex}% space after theorem head; " " = normal interword space
  {}% Manually specify head
\theoremstyle{homework} 

% Set up an Exercise environment and a Solution label.
\newtheorem*{exercisecore}{Exercise \@currentlabel}
\newenvironment{exercise}[1]
{\def\@currentlabel{#1}\exercisecore}
{\endexercisecore}

\newcommand{\localhead}[1]{\par\smallskip\noindent\textbf{#1}\nobreak\\}%
\newcommand\solution{\localhead{Solution:}}

%%%%%%%%%%%%%%%%%%%%%%%%%%%%%%%%%%%%%%%%%%%%%%%%%%%%%%%%%%%%%%%%%%%%%%%%
%
% Stuff for getting the name/document date/title across the header
\makeatletter
\RequirePackage{fancyhdr}
\pagestyle{fancy}
\fancyfoot[C]{\ifnum \value{page} > 1\relax\thepage\fi}
\fancyhead[L]{\ifx\@doclabel\@empty\else\@doclabel\fi}
\fancyhead[C]{\ifx\@docdate\@empty\else\@docdate\fi}
\fancyhead[R]{\ifx\@docauthor\@empty\else\@docauthor\fi}
\headheight 15pt

\def\doclabel#1{\gdef\@doclabel{#1}}
\doclabel{Use {\tt\textbackslash doclabel\{MY LABEL\}}.}
\def\docdate#1{\gdef\@docdate{#1}}
\docdate{Use {\tt\textbackslash docdate\{MY DATE\}}.}
\def\docauthor#1{\gdef\@docauthor{#1}}
\docauthor{Use {\tt\textbackslash docauthor\{MY NAME\}}.}
\makeatother

% Shortcuts for blackboard bold number sets (reals, integers, etc.)
\newcommand{\Reals}{\ensuremath{\mathbb R}}
\newcommand{\Nats}{\ensuremath{\mathbb N}}
\newcommand{\Ints}{\ensuremath{\mathbb Z}}
\newcommand{\Rats}{\ensuremath{\mathbb Q}}
\newcommand{\Cplx}{\ensuremath{\mathbb C}}
%% Some equivalents that some people may prefer.
\let\RR\Reals
\let\NN\Nats
\let\II\Ints
\let\CC\Cplx

%%%%%%%%%%%%%%%%%%%%%%%%%%%%%%%%%%%%%%%%%%%%%%%%%%%%%%%%%%%%%%%%%%%%%%%%%%%%%%%%%%%%%%%
%%%%%%%%%%%%%%%%%%%%%%%%%%%%%%%%%%%%%%%%%%%%%%%%%%%%%%%%%%%%%%%%%%%%%%%%%%%%%%%%%%%%%%%
% 
% The main document start here.

% The following commands set up the material that appears in the header.
\doclabel{STAT 402: Homework 7}
\docauthor{Stefano Fochesatto}
\docdate{\today}


%\textbf{Code:}
%\begin{center}
 %   \lstinputlisting{r1.txt}
%\end{center}

\begin{document}

\begin{exercise}{1}We want to know what proportion of ponds in a region have fish in them. However, 
  the ponds occur in groups so that if we visit one pond we ought to just sample all ponds in the 
  group. So we divide (using an aerial photo) the area into $N = 100$ clusters, each consisting of 
  from one to five ponds. We will select (SRS) $n = 10$ of the clusters. We get the following results:\\

\noindent * Cluster 1: $\tau_1 = 3$ ponds have fish, out of $M_1 = 4$ ponds.\\
* Cluster 2: $\tau_2 = 2$ ponds have fish, out of $M_2 = 3$ ponds.  \\
* Cluster 3: $\tau_3 = 1$ pond has fish, out of $M_3 = 4$ ponds. \\
* Cluster 4: $\tau_4 = 1$ pond has fish, out of $M_4 = 3$ ponds. \\
* Cluster 5: $\tau_5 = 3$ ponds have fish, out of $M_5 = 5$ ponds. \\
* Cluster 6: $\tau_6 = 2$ ponds have fish, out of $M_6 = 4$ ponds. \\
* Cluster 7: $\tau_7 = 4$ ponds have fish, out of $M_7 = 5$ ponds. \\
* Cluster 8: $\tau_8 = 2$ ponds have fish, out of $M_8 = 2$ ponds. \\
* Cluster 9: $\tau_9 = 4$ ponds have fish, out of $M_9 = 5$ ponds. \\
* Cluster 10: $\tau_{10} = 5$ ponds have fish, out of $M_{10} = 5$ ponds.\\

Find a 95 percent confidence interval for the proportion of ponds that have fish, in the region.\\
\solution Since we do not know the total number of ponds in the in the region (no $M$) we have to use the 
one stage cluster sample ratio estimator to compute the proportion. Using the ratio estimator in r we get,\\ 
\textbf{Code:}
\begin{center}
   \lstinputlisting{r1.txt}
\end{center}
\end{exercise}
\vspace{1in}


\begin{exercise}{2}Read the paper Mortality Before and After the 2003 Invasion of Iraq (lancet.pdf). 
  How did they conduct the sample? Why was cluster sampling the only reasonable way to conduct this survey?\\
  \solution The paper describes a modified two-stage cluster sample. There were 33 clusters and in each cluster there sampled 30
  households. The 33 clusters where randomly divided among the 18 governorates, then a second stage of reassignment was used to 
  group the clusters in 6 pairs of governorates for safety and travel reasons. Without conducting an entire census the two-stage cluster 
  sample gave the researchers the flexibility to actually conduct the experiment. Begin able to reassign clusters for safety, and take a reasonable sample 
  in each cluster made this study possible. 
\end{exercise}
\vspace{1in}




\begin{exercise}{3} We want to know the proportion of nests in a forest that are in current use. 
  What we’ll do is divide the forest into $N = 10000$ 20m by 20m plots and count the number 
  of nests $M_i$ in each of $n$ plots that we obtain from a simple random sample of plots. 
  Once we get our n plots, we’ll select (SRS) $m_i$ nests in the ith plot and examine these. 
  The number being used is $\tau_i$. This is similar to problem 1, but differs in one crucial way.

  Plot $M_i$ $m_i$ $\tau_i$
  
  1, 52, 20, 12 
  
  2, 40, 20, 11 
  
  3, 35, 20, 15 
  
  4, 38, 20, 5 
  
  5, 51, 25, 12
  
  6,14, 20, 8
  
  7, 50, 25, 13 
  
  8, 42, 20, 9 
  
  9, 37, 20, 7 
  
  10, 30, 20, 14 
  
  11, 22, 20, 16 
  
  12, 12, 10, 4
  
\begin{enumerate}
  \item[a.]Find a 95 percent confidence interval for the true proportion p.\\
  \solution This problem is similar to the first since we will be using the ratio cluster sample estimator but 
  this time we are working with a two-stage cluster sample. Computing the two-stage cluster sample estimator in r we get,\\
  \textbf{Code:}
\begin{center}
   \lstinputlisting{r2.txt}
\end{center}




  \item[b.]When would you sample fewer clusters (plots) and more inside each cluster (plot)?\\
  \solution We can sample fewer clusters and more inside each clusters when we know there is little variance 
  between clusters and more variance inside of each cluster. This can occur if clusters in a sample vary wildly, but each cluster seems 
  to exhibit the same behavior. Suppose we are estimating grass cover and each cluster exhibits the same level of patchiness, and the sample in each 
  cluster have high variance. We can see this by computing the terms in the variance independently, in our case the cluster to cluster variability is a lot higher
  so we would want to visit more clusters. In fact the variability inside each cluster seems very low and it might be better to sample less in each cluster.
\end{enumerate} 
\end{exercise}
\vspace{1in}



\begin{exercise}{4} 
  Here's another two-stage cluster sampling problem.
  
  Suppose I want to know the total value of an inventory. There are $N = 30$ warehouses, 
  and I don’t want to go to all of them. Also, each warehouse is too large to sample 
  everything, though the items (should) be somewhat similar in value. I’ll select 
  $n = 5$ warehouses as a SRS, then I’ll count the number of items ($M_i$), take a 
  sample of $m_i = 20$ items from each warehouse, then get an average value and standard 
  deviation from each warehouse:\\


\noindent* warehouse 1: $M_1 = 2100$, $m_1 = 20$, $\overline{x}_1 = \$35.00$, $s_1 = \$10.00$. \\
         * warehouse 2: $M_2 = 850$, $m_2 = 20$, $\overline{x}_2 = \$42.00$, $s_2 = \$12.00$. \\
         * warehouse 3: $M_3 = 1500$, $m_3 = 20$, $\overline{x}_3 = \$40.00$, $s_3 = \$8.00$. \\
         * warehouse 4: $M_4 = 2200$, $m_4 = 20$, $\overline{x}_4 = \$38.00$, $s_4 = \$5.00$. \\
         * warehouse 5: $M_5 = 500$, $m_5 = 20$, $\overline{x}_5 = \$18.00$, $s_5 = \$6.00$.\\
  
\begin{enumerate}
  \item[a/b.]  Find an estimate for the total value of items in all warehouses and obtain a 95 percent confidence interval. 
  Note that we only get an item count for warehouses we sample.\\
  \solution Since we don't have the total number of items in the inventory $M$ we will use the unbiased estimator of the total. 
  Computing the estimator in r we get,\\
  \textbf{Code:}
  \begin{center}
     \lstinputlisting{r3.txt}
  \end{center}

\end{enumerate}



\end{exercise}




\end{document}




















